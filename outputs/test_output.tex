\documentclass{article}
\usepackage[utf8]{inputenc}
\usepackage{graphicx}
\usepackage{amsmath}
\usepackage{amssymb}
\usepackage{enumitem}
\usepackage{booktabs}
\usepackage{hyperref}

\title{Project Proposal INFO 290T}
\author{Luca Charrouf}
\date{}

\begin{document}

\maketitle

\section*{Dataset}
Skin Cancer MNIST - HAM10000: This dataset consists of 10,000 dermatoscopic images of
pigmented skin lesions across seven different diagnostic categories. The dataset was
curated to train neural networks for automated diagnosis of pigmented skin lesions.

\subsection*{Categories in dataset:}
\begin{itemize}
    \item Actinic keratoses and intraepithelial carcinoma
    \item Basal cell carcinoma
    \item Benign keratosis-like lesions
    \item Dermatofibroma
    \item Melanoma
    \item Melanocytic nevi
    \item Vascular lesions
\end{itemize}

\subsection*{Data attributes:}
\begin{itemize}
    \item Image files in HAM10000\_images
    \item Metadata including age, sex, localization, and diagnostic confirmation in
    HAM10000\_metadata
\end{itemize}

\section*{Data Preprocessing}
\begin{itemize}
    \item \textbf{Image Resizing:} Standardize all images to the same dimensions to ensure
    consistent feature extraction.
    \item \textbf{Color Normalization:} Standardize color representation to reduce variations in
    lighting and camera settings, which are common in dermatoscopic imagery.
    \item \textbf{Duplicates:} The dataset contains multiple images of the same lesion (with same
    lesion\_id but different image\_id)
\end{itemize}

\section*{Classification Proposal}
The project aims to build a model that can accurately classify dermatoscopic images into
their corresponding skin lesion categories. This classification has significant real-world
applications in assisting dermatologists with preliminary diagnosis and screening.

\section*{Intended Features}
\begin{itemize}
    \item \textbf{Histogram of Oriented Gradients (HOG):} To capture edge and texture patterns that
    are crucial for distinguishing between different types of skin lesions.
    \item \textbf{Color Histograms:} To quantify color distributions, as color is a key diagnostic feature
    in dermatology (e.g., melanomas often have specific color patterns).
    \item \textbf{Shape Descriptors:} To quantify border irregularity and asymmetry, which are
    important clinical indicators in skin cancer diagnosis.
\end{itemize}

\section*{Classifiers}
\begin{itemize}
    \item \textbf{Convolutional Neural Networks (CNNs):} This will be most likely the most accurate
    classifier, given that they are specifically designed for image processing tasks.
    \item \textbf{Random forests:} Provides robust classification with feature importance metrics that
    can help identify which image characteristics are most diagnostically relevant.
\end{itemize}

\section*{Evaluation Metrics}
Given the medical nature of this classification task and likely class imbalance, we'll focus on:
\begin{itemize}
    \item Precision, recall, and F1-score per class
    \item Confusion matrix analysis
    \item ROC curves and AUC metrics
    \item Sensitivity and specificity, particularly for malignant vs. benign classifications
\end{itemize}

\end{document}